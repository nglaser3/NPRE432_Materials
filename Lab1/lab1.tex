\documentclass{article}

% Language setting
% Replace `english' with e.g. `spanish' to change the document language
\usepackage{biblatex} %Imports biblatex package
\addbibresource{refs.bib}
\usepackage[english]{babel}
\usepackage{array}
\usepackage{amsmath}
\usepackage{pythonhighlight}
\usepackage{multirow}
\newcolumntype{P}[1]{>{\centering\arraybackslash}p{#1}}
\newcolumntype{M}[1]{>{\centering\arraybackslash}m{#1}}

% Set page size and margins
% Replace `letterpaper' with `a4paper' for UK/EU standard size
\usepackage[letterpaper,top=2cm,bottom=2cm,left=3cm,right=3cm,marginparwidth=1.75cm]{geometry}

\usepackage{amsmath}
\usepackage{graphicx}
\usepackage[colorlinks=true, allcolors=blue]{hyperref}
\usepackage{setspace}
\usepackage{booktabs}
\usepackage[T1]{fontenc}
\usepackage{longtable}
\doublespacing

\begin{document}
\begin{titlepage}

\centering
\scshape
\vspace{\baselineskip}

%
\rule{\textwidth}{1.6pt}\vspace*{-\baselineskip}\vspace*{2pt}
\rule{\textwidth}{0.4pt}

{\Huge \textbf{\textsc{ Hardness and Compression \\
\vspace{15pt}}}}

\rule{\textwidth}{0.4pt}\vspace*{-\baselineskip}\vspace{3.2pt}
\rule{\textwidth}{1.6pt}\vspace{6pt}
\centerline{\textit{University of Illinois at Urbana-Champaign}} 
\centerline{\textit{Department of Nuclear, Plasma, and Radiological Engineering}}
\vspace{1.5\baselineskip}


\large \centerline{\textbf{Author:} Nathan Glaser}
\large \centerline{\textbf{Net-ID:} nglaser3}
\quad

\vfill
\large \centerline{September 18, 2024}
%
\pagenumbering{gobble}
\end{titlepage}

\tableofcontents
\newpage
\pagenumbering{arabic}

\section{Abstract}

Materials, especially in nuclear applications, may encounter extreme external forces. To predict how materials will fare in these scenarios we must have an understanding of its deformation tendencies (plasticity and elasticity) as a function of external load and its resistance to these deformations (hardness). We investigated four materials for their hardness and behaviour under compression: PolyMethyl MethAcrylate (PMMA), 1018 Cold Rolled (CR) steel, 1045 NorMalized (NM) Steel, and 2024 Aluminum alloy. To measure their hardness, we subjected each material (excluding PMMA) to three trials of Rockwell B/C hardness tests. To examine their behaviour under compression we utilized an INSTRON universal testing machine and compressed each of the four materials. Further, we examined the difference in efficacy of three hardness test (Brinnel, Rockwell B/C) on two materials: 2024 Aluminum alloy and 4340 carbon steel. We found that the reliability of the various hardness tests is strongly dependent on the material being tested, and that the conversions from Rockwell B to Brinnel was more inaccurate than from Rockwell C to Brinnel, though it should be noted that the Rockwell C value was outside of the validity range for the conversion, and so we then converted to Vickers instead. 

\newpage
\section{Results}
To begin, we measured the hardness of 2024 aluminum alloy and 4340 carbon steel using Brinnel and Rockwell B/C tests. 20 trials were conducted and tabulated. The statistical information of these trials is presented in Fig. \ref{tab:q1}. Br is Brinnel, and Rb/Rc are Rockwell B/C respectively. 

\begin{table}[hp]
    \def\arraystretch{1.5}
    \centering
    \caption{Hardness Test Statistics for Aluminum Alloy 2024 and Carbon Steel 4340}
    \label{table:q1}
    \begin{tabular}{|c|c|c|c|c|c|c|}
    \toprule
    \hline
    \multicolumn{1}{|c|}{\textbf{Material:}} & \multicolumn{3}{|c|}{\textbf{2024 Aluminum}} & \multicolumn{3}{|c|}{\textbf{4340 Steel}} \\ \hline
    \textbf{Test:} & \textbf{Brinell} & \textbf{Rockwell B} & \textbf{Rockwell C} & \textbf{Brinell} & \textbf{Rockwell B} & \textbf{Rockwell C} \\ \hline
%    \midrule
    \textbf{Mean} & 140.9 & 86.475 & 4.115 & 740.5 & 121.485 & 63.04 \\
    \textbf{Median} & 138. & 86.65 & 5.2 & 731. & 121.55 & 62.85 \\
    \textbf{Std. Dev.} & 6.172 & 2.927 & 3.033 & 38.124 & 0.977 & 1.495 \\
    \hline
\end{tabular}
\end{table} 

To continue, we converted whichever of the Rockwell B/C tests that yielded sensible results to Brinnel utilizing Table 5 and 6 from \cite{manual}. We found the Rockwell B test for 2024 was sensible, and converted to a Brinnel value of 170.95. Further, we found the Rockwell C test for 4340 was sensible, however was outside of the range of validity of the conversion chart presented in Table 6 from \cite{manual}. Thus, we instead converted to Vickers, obtaining a value of 768.1.

To continue, 


\section{Bibliography}
\printbibliography[heading=none]

\end{document}